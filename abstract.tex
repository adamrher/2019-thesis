Many next-generation atmospheric general circulation models (AGCM) allow for substantial grid flexibility, enabling the representation of a wider-range of horizontal scales at reduced computational cost through the use of variable resolution grids. AGCMs, however, are notoriously sensitive to grid resolution, and in some cases, solutions noticeably diverge across the refinement region of variable resolution grids. The lack of scientific consensus on the reason(s) solutions diverge with increasing horizontal resolution remains an obstacle to more widespread application of variable resolution grids to regional climate problems. It is the purpose of this thesis to develop and apply a hierarchy of reduced complexity model configurations across horizontal resolutions, to isolate and understand resolution sensitivity in an AGCM, the Community Atmosphere Model (CAM). The hierarchy of configurations used in this thesis range from a dry thermal bubble test, to an aqua-planet configuration with comprehensive moist physics, to a realistic, present day simulation with regional grid refinement over the big ice sheet in Greenland. The magnitude of resolved vertical motion increases substantially with resolution in response to the buoyancy produced by the stratiform cloud scheme. This sensitivity follows from the fact that higher resolution grids are able to support smaller buoyancy length scales, forcing pressure gradients and related vertical velocities to scale like the inverse of the grid spacing.

An understanding of resolution sensitivity has proven useful in two separate applications. (1) The impact of computing the physical parameterizations on a separate, coarser resolution grid than used by the dynamical core (a `physics grid') is evaluated in detail. It was found that a coarser physics grid does not degrade the effective resolution of the model, with the vertical velocity still determined by the dynamical core grid spacing, rather than the physics grid spacing. Since the physical parameterizations are responsible for about half the computational cost of the conventional CAM configuration, the coarser physics grid allows for significant cost savings with little to no downside. (2) High-latitude regions are less sensitive to resolution due to relatively subdued buoyancy induced vertical velocities, and justifies the use of variable resolution grids to study the climate of polar ice sheets. A modeling study with grid refinement over the Greenland Ice Sheet has shown to alleviate a long-standing positive precipitation bias in the ice sheet interior in CAM. The culmination of this thesis is a robust framework for understanding resolution sensitivity, which may be used to anticipate and alleviate issues arising from the increasingly diverse types of grids that are now common among AGCMs for simulating regional climate.