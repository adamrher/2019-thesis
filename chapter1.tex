\subsection{A brief history of Atmospheric General Circulation Models}

The atmosphere consists of a large number of moving parts, operating over a massive range of space-time scales. In the 1950's, scientist Norman Phillips noted that  ``dishpan" experiments --- a rotating pan of water heated on its edges with an electric coil --- appear remarkably similar to large-scale weather patterns in the atmosphere \citep{WEART2008},
\begin{displayquote}
{\em{at least the gross features of the general circulation of the atmosphere can be predicted without having to specify the heating and cooling in great detail.}}
\end{displayquote}
Inspired by this observation, Phillips developed a 2-layer numerical model of the atmosphere using a simplified set of equations and idealized heating of the lower layer. His model produced a realistic looking jet-stream with synoptic disturbances. Like the dishpan experiments, his numerical model was able to produce the large-scale features of the atmosphere while neglecting finer-scale processes thought to be important in the atmosphere. The Phillips model is considered the first Atmosphere General Circulation Model (AGCM); it used a horizontal grid spacing $\Delta x = 22.5^{\circ}$.

Phillips' AGCM only applied to the mid-latitudes; models developed by \cite{METAL1965MWR, HH1980JAS} solidified the idea that relatively simple parameterizations of diabatic heating can produce realistic tropical circulations. The study of \citep{HH1980JAS} is in particular quite remarkeable in that it does not contain any explicit representation of moisture. Moist processes and radiative cooling are parameterized as a simple newtonian relaxation of potential temperature to a modestly unstable reference profile. Despite the crudeness of this approximation, a realistic Hadley Cell is is simulated. 

\cite{METAL1965MWR} explicitly simulates advection of water vapor by the resolved dynamics, parameterizes convection and clouds using a critical lapse rate approach and super-saturation criteria, and radiation using a radiative transfer model. The \cite{METAL1965MWR} AGCM is significantly more complex than any AGCM that had come before, and is structurally similar to modern AGCMs. The horizontal grid spacing of the Manabe et al. AGCM is $\Delta x = 4.5^{\circ}$

This brief history of AGCM development is intended to provide the reader with the reason AGCMs are ran on grids that cannot resolve and therefore largely ignore a myriad of scales and processes. The short answer --- because they have no choice and it works. But with the vastly greater computational power realized over the decades following the first AGCM models, scientists have pushed for finer and finer grid resolutions, and with these, a whole new set of challenges. The title of this thesis --- understanding resolution sensitivity in the Community Atmosphere Model --- is focused on one of these problems in one AGCM. Standard convergence tests in the Community Atmosphere Model (CAM) and it's predecessor versions result in weak, or non-convergent solutions with increasing horizontal resolution \citep{KW1991JGR,WETAL1995CD,W1999T,W2008TELLUS,LETAL2011TELLUS,RJ2011MWR,RETAL2012ASL,OETAL2013JCLIM,RETAL2013JCLIM,ZetAl2014JCb,LETAL2015JCLIM}. While this has been a documented feature of the model since it's inception, from a strict computational fluid dynamics perspective this is not acceptable model behavior \citep{W2008TELLUS}. The following sections provide a literature review of these convergence studies, which span about about thirty years.

\subsection{CCM1}

The National Center for Research (NCAR) Community Climate Model, version 1 \citep{CCM1} is the successor to CCM0 and \cite{KW1991JGR} provides the first convergence study using a NCAR climate model. The CCM1 dynamical core is built with the global-spectra transform method. Explicit numerical dissipation is occurs through biharmonic $\nabla^{4}$ diffusion, and a moisture corrective factor $q-flux$ is required to compensate for clipping of negative water vapor arising from oscillatory noise typical of spectral method. The model consists of 12 vertical levels using a $\sigma$ coordinate. The parameterization of clouds in CCM1 is broadly similar to \citep{METAL1965MWR}, and depends on lapse rate and grid cell relative humidity. A stable stratiform cloud forms with cloud fraction $0.95$  when a grid cell is super-saturated and the lapse-rate is stable. If the lapse rate is steeper than the moist adiabat, the total cloud fraction of the unstable layer is set to $0.3$, and the level above the unstable layer is set to $0.95$, approximating an anvil cloud top. Radiative transfer is parameterized after \cite{CCM1RAD}. A parameterization of momentum flux divergence due to stationary gravity waves is included \citep{M1987JAS}. 

 \begin{table}
 \caption{\cite{KW1991JGR} experimental design.}
 \centering
 \begin{tabular}{llcccc}
 \hline
 Variable & $4.5^{\circ}$ (R15) & $2.8^{\circ}$ (T42) & $1.9^{\circ}$ (T63)  & $1.1^{\circ}$ (T106) \\
 \hline
   $\nabla^{4}$ coefficient ($m^4/s$) & $2.0 \times 10^{16}$ & $1.0 \times 10^{16}$ & $2.0 \times 10^{15}$ & $1.0 \times 10^{15}$ \\
   $\Delta t_{phys}$ (s) & 1800 & 900 & 600 & 360 \\
   Total Cloud Fraction & 0.47 & 0.36 & 0.29 & 0.26 \\
   Total Precipitable Water (mm) & 21.1 & 20.0 & 18.9 & 18.8 \\
 \hline
 \end{tabular}
 \label{tbl:table1-1}
 \end{table}

The convergence study of \cite{KW1991JGR} consists of an ensemble of simulated January's. Considering the computational limitations at the time, an impressive range of resolutions are covered in this study (Table~\ref{tbl:table1-1}). The boundary conditions contain real-world topography, and prescribed seasonally varying sea-surface temperatures (SST). The only parameters that vary with resolution are the physics time-step, $\Delta t_{phys}$, the standard deviation of the sub-grid topography (for the gravity wave drag parameterization), and the biharmonic diffusion coefficient for numerical dissipation. The coefficients of biharmonic diffusion were selected to preserving the slope of kinetic energy spectrum \citep[see][]{B1991JCLIM}. The lowest resolution simulation (R15) statistics are found by averaging the January data from 11 simulated annual cycles, and data from the next highest resolution (T42) are computed from 6 annual cycles. In the higher resolution simulations (T63 and T106), three simulations each were initialized from Dec. 15th of the T42 ensembles, and integrated through the month of January.

Table~\ref{tbl:table1-1} provides the global mean total cloud fraction and total precipitable water from the four simulations. With increasing resolution, the model generally becomes less cloudy and drier, although one should probably take precaution interpreting the highest resolution runs, since the values are computed from a limited sample. \cite{KW1991JGR} show that the magnitude of the upward and downward vertical motion increases with resolution, and argues the greater subsidence is responsible for the drier and less cloudy atmosphere at higher resolutions. In the zonal mean, it was discussed that the width of ascending branch of the Hadley Cell was found to be grid limited at the lower resolutions, containing only $3-4$ grid cells in the R15 case, but that at the higher resolutions (T63 and T106) the width had converged and reached the ``natural scale of the phenomena."

\subsection{CCM2}

CCM, version 2 (CCM2) is described in \cite{CCM2} and the convergence study in \cite{WETAL1995CD}. CCM2 uses the spectral-transform dynamical core from CCM1, but water vapor is transported using a semi-Lagrangian method \citep{WR1994TELLUS}, since oscillatory errors from the spectral method in CCM1 too often drove water vapor concentrations negative. CCM2 contains 18 vertical levels, and uses a hybrid-$\sigma$ vertical coordinate. The physical parameterizations have undergone a major overhaul. CCM2, unlike its predecessor, includes a diurnal cycle and atmospheric absorption of solar radiation. Convection is handled through a mass flux scheme \citep{H1994JGR}. The planetary boundary layer scheme is described in \cite{HB1993JCLIM} and includes non-local, up-gradient mixing. The stratiform cloud scheme has added dependencies on vertical motion and static stability, and introduces a height varying relative humidity threshold to distinguish mid-to-high level stratus from marine stratocumulus clouds \citep{KETAL1994JGR}.

\cite{WETAL1995CD} is unique in that consists of two convergence tests, one with fixed cloud parameters across resolutions (``unmodified parameterizations"), and one that contains resolution dependent cloud parameters to minimize changes in total cloud fraction with resolution (``modified parameterizations"). The impact of drying at higher resolution in CCM1 (Table~\ref{tbl:table1-1}) may be avoided through decreasing the relative humidity thresholds in the cloud parameterization \citep{KETAL1994JGR} to reduce the dependency of cloud fraction on resolution. They were fairly successful; after tuning, total cloud fraction only varied from $0.58$ at low resolution (R15) to $0.55$ at high resolution (T106; compare to Table~\ref{tbl:table1-1}).

 \begin{table}
 \caption{\cite{WETAL1995CD} experimental design.}
 \centering
 \begin{tabular}{llccc}
 \hline
 Variable & $3.75^{\circ}$ (T31) & $2.8^{\circ}$ (T42) & $1.9^{\circ}$ (T63)  & $1.1^{\circ}$ (T106) \\
 \hline
   $\nabla^{4}$ coefficient ($m^4/s$) & $2.0 \times 10^{16}$ & $1.0 \times 10^{16}$ & $2.0 \times 10^{15}$ & $1.0 \times 10^{15}$ \\
   $\Delta t_{phys}$ (s) & 450 & 450 & 450 & 450 \\
   Total Cloud Fraction & 0.56 & 0.54 & 0.52 & 0.50 \\
   Total Precipitable Water (mm) & 24.3 & 23.9 & 22.6 & 22.5 \\
   Convective Precipitation (mm/day) & 2.62 & 2.58 & 2.56 & 2.55 \\
   Stratiform Precipitation (mm/day) & 0.83 & 0.89 & 0.93 & 1.08 \\   
 \hline
 \end{tabular}
 \label{tbl:table1-2}
 \end{table}

Table~\ref{tbl:table1-2} shows the results of the convergence tests in \cite{WETAL1995CD} in the unmodified parameterizations simulations. I emphasize the unmodified simulations since this thesis aims to understand why solutions change with resolution, rather than mitigate changes in resolution through tuning. The statistics refer to January means, and are directly comparable to the CC1 convergence test (Table~\ref{tbl:table1-1}). The rationale for fixing the physics time-step is articulated more clearly in \cite{W2013QJRMS}, but arises from a competition between the convective scheme \citep[i.e.,;][]{H1994JGR}, which remove instability at a specified rate, with stratiform cloud schemes, which remove instability instantaneously. Similar to CCM1, cloud fraction and total precipitable water decrease with resolution. The convective precipitation rate decreases with resolution while stratiform precipitation increases with resolution.

\subsection{CAM3}
\subsection{CAM4}
\subsection{CAM5}